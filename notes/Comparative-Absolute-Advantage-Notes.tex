\documentclass[onehalfspacing,11pt]{article}
\usepackage{achicago,setspace,palatino,graphicx}
\usepackage{amsmath,amssymb}
\usepackage{bm,bbm}
\usepackage[bottom]{footmisc} % places footnotes @ bottom of page
\usepackage{booktabs}
\usepackage{enumerate}
\usepackage[breaklinks]{hyperref} % option enables "broken" links across lines
\usepackage{natbib}
\usepackage[usenames,dvipsnames]{pstricks}
\usepackage{epsfig}
\usepackage{caption}
\usepackage{subcaption}
\usepackage{lscape,rotating}
\usepackage{nicefrac}
\usepackage{dsfont}
\usepackage{mathtools}

\DeclarePairedDelimiter\abs{\lvert}{\rvert}%
\DeclarePairedDelimiter\norm{\lVert}{\rVert}%

% Swap the definition of \abs* and \norm*, so that \abs
% and \norm resizes the size of the brackets, and the 
% starred version does not.
\makeatletter
\let\oldabs\abs
\def\abs{\@ifstar{\oldabs}{\oldabs*}}
%
\let\oldnorm\norm
\def\norm{\@ifstar{\oldnorm}{\oldnorm*}}
\makeatother

%\newcommand*{\Value}{\frac{1}{2}x^2}%
%\begin{document}
%    \[\abs{\Value}  \quad \norm{\Value}  \qquad\text{non-starred}  \]
%    \[\abs*{\Value} \quad \norm*{\Value} \qquad\text{starred}\qquad\]
%\end{document}

\newtheorem{as}{Assumption}
\newtheorem{conjecture}{Conjecture}
\newtheorem{corr}{Corollary}
\newtheorem{df}{Definition}
\newtheorem{lemma}{Lemma}
\newtheorem{prp}{Proposition}
\newtheorem{clm}{Claim}
\newtheorem{rmk}{Remark}
\newenvironment{prf}{{\bf Proof}}{\hfill {\sc q.e.d. }}
\newenvironment{prfLemma}{{\bf Proof of Lemma}}{\hfill {\sc q.e.d. }(Lemma)}
\setlength{\parindent}{0em} \setlength{\parskip}{1.5ex plus0.5ex
minus0.5ex} \textwidth15.75cm \evensidemargin5mm \oddsidemargin5mm
\topmargin-8mm \textheight 21.7cm

\newcommand{\fraction}{\int\frac{\mu_{t+1}}{\epsilon_{t+1}} }
\newcommand{\lbar}{\int\frac{\mu_{t+1}}{\epsilon_{t+1}} }
\parindent 0pt
\parskip 5pt
\def\newblock{\hskip .11em plus .33em minus .07em}
\hypersetup{colorlinks=true,urlcolor=blue,linkcolor=blue,citecolor=blue}

%%%%%%%%%%%%%%%%%%%%%%%%%%%%%%%%%%%%%%%%%%%%%%%%%%%%%%%%%%%%%%%%%%%%%

\begin{document}

%\begin{titlepage}
%\begin{singlespacing}

\title{Comparative and Absolute Advantage:\\%\footnote{}}
Ability in the Data and in the Model}

%\author{Simeon D.~Alder\footnote{University of Wisconsin - Madison, Department of Economics, email: \url{sdalder@wisc.edu}} \and CO-AUTHOR}

\date{\today \\ \vspace{5mm} }%{\sc Preliminary and Incomplete -- Please Do Not Cite}}

\maketitle

%\begin{abstract}
%ABSTRACT
%\end{abstract}
%\noindent
%\textit{JEL Codes:}
%
%\textit{Keywords:}
%\end{singlespacing}
%\end{titlepage}
In Project TALENT and in the two iterations of the NLSY, each survey respondent takes a battery of cognitive tests and we have test scores for three types of tests:
\begin{enumerate}
  \item mathematics,
  \item verbal (word knowledge and comprehension), and
  \item social.
\end{enumerate}
O*NET has skill requirements for each occupation and we are focusing on the skills corresponding most closely to the cognitive test scores, namely:
\begin{enumerate}
  \item mathematics,
  \item writing, and
  \item social skills.
\end{enumerate}
We'll label the three components $m$, $w$, and $s$, respectively.

Each individual is characterized by a triple $(a_m,a_w,a_s)$ based on her test scores, which are normalized to the unit interval $[0,1]$ to facilitate comparisons across cohorts in different surveys. Effectively, the triple keeps track of an individual's rank in each individual component.

Similarly, each occupation $j$ is characterized by a vector of skill requirements $(b_m,b_w,b_s)$ based on the O*NET descriptors.

Jobs are differentiated vertically as well as horizontally. For instance, some occupations have higher requirements in all three dimensions compared to other jobs. They require more skill in an {\it absolute} sense. Other jobs have a different mix of requirements. For instance, one occupation may have low requirements for mathematics and high requirements for writing / communication while another job may have a ``reverse'' profile. This differentiation is mostly horizontal and hence the requirements vary in a {\it comparative} sense.

So how can we assign occupation-specific abilities to each individual with the test scores we have? In a nutshell, we are proposing a distance measure between an individual's test score profile and the skill requirement of each occupation that takes into account absolute as well as comparative differentiation.

Without loss of generality (at least I think so), the mathematical skill and requirement pair is the reference dimension in our measure of distance between individual $i$'s skills and occupation $j$'s requirements.
\begin{equation}
\label{ }
\hat{a}_i^j = \ln\left( \frac{a_{i,m}}{b_{j,m}} \right) - \abs{ \ln \left( \frac{\nicefrac{a_{i,w}}{b_{j,w}}}{\nicefrac{a_{i,m}}{b_{j,m}}}  \right) } - \abs{ \ln \left( \frac{\nicefrac{a_{i,s}}{b_{j,s}}}{\nicefrac{a_{i,m}}{b_{j,m}}}  \right) }
\end{equation}
This measures the distance between an individual's skill set relative to the requirements of a particular occupation. The support of this measure is the real line.

For the purposes of our quantitative exercise, we need to transform this measure such that the support is $\mathbb{R}^{+} \cup \{ 0 \}$ by subtracting the minimum of all $\hat{a}_i^j $. In addition, we divide it by the standard deviation of $\hat{a}_i^j $ in order to standardize it.

\begin{equation}
\label{ }
\overline{a}_i^j = \frac{\hat{a}_i^j - \min\{\hat{a}_i^j\}_{i,j} }{\sigma(\hat{a}_i^j)}
\end{equation}
Each individual in this economy now is endowed with a vector of occupation-specific abilities $\overline a_i = \{ \overline a_i^j \}_{j \in J}$, where $J$ is the number of different occupations in the economy.

In the non-linear occupational choice environment of our model, both absolute and comparative advantage matter, which is why $\overline a_i$ takes into account both aspects.
\end{document}